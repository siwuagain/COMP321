\documentclass[11pt]{article}
\usepackage{amsmath, amssymb}
\usepackage{hyperref}
\usepackage{graphicx}
\usepackage{multicol}

% Define \problemname and \illustration if not using Kattis template
\newcommand{\problemname}[1]{\section*{#1}}
\newcommand{\illustration}[3]{%
    \begin{figure}[h]
        \centering
        \includegraphics[width=#1\textwidth]{#2}
        \caption{#3}
    \end{figure}
}

\begin{document}

\problemname{Booby Trap Treasure Hunt}

% Example use of the \illustration command
% \illustration{0.3}{filename.jpg}{Photo by \href{url_here}{link text here, e.g. author name}}

% Put the problem story here. Here are some common issues to keep in mind as you edit:

% \begin{itemize}
%     \item Use \LaTeX{} math mode for (most) numbers. The few exceptions are for years (e.g., 2019) or whenever a number is part of a string token.
%     \item For numbers $1\,000$ or larger, use thousands separator \verb|\,|. So, for example, one million would be $1\,000\,000$.
%     \item Write in active voice (i.e., avoid passive voice).
%     \item Write in present tense (avoid the word \texttt{will}, especially in the input description). For example, prefer: ``Input starts with an integer $n$ followed by $n$ lines...'' rather than ``Input \emph{will} start with an integer $n$, then come $n$ lines...''.
% \end{itemize}

Jimothy is a seasoned adventurer scavenging the underground golden dungeon of the mad wizard. He stumbles onto a room where $N$ traps of increasing complexity guard the way.\\\\
Jimothy needs to disarm every \(trap_i\) but has to warm up by solving the easiest (\(trap_1\)) before he gets to choose, otherwise he does not dare disarm the others.\\\\
Each \(trap_i\) requires \(S_i\) steps and has \(C_i\) complexity. For each step of disarming a trap, Jimothy incurs fatigue equal to the complexity of the trap he has solved with the most steps (zero if it is the first).\\\\
The most difficult trap contains an artifact that allows him to disarm all the other ones, allowing for easy treasure collection! How greedy...\\\\
Help Jimothy minimize his fatigue after disarming all the traps so that he can keep exploring this never-ending trove.
\begin{figure}[!h]
\includegraphics[width=1\textwidth]{321_project_figure_2.png}
\caption{Example}
\label{fig:trap}
\end{figure}
\section*{Input}
% Put the input format (syntax) description here, including constraints on all input values. That includes number ranges, the maximum number of digits after the decimal point for real values, the character sets and length ranges for strings, etc. If possible, always give a count of the number of items that follow (test cases, e.g.) rather than delimiting with a token or EOF. The input format should be fairly precise, so that the reader knows exactly what to expect (it should usually be easy to write a program to read the input using line-oriented or token-oriented parsing).
The first line of input contains an integer \(1 \le N \le 10^{5}\) indicating the number of traps. 
The next $N$ lines each represent a \(trap_i\), where $i$ ranges from 1 to $N$. 
Each of these lines contains two integers: \(1 \le S_{i} \le 10^{5}\), which denotes the number of steps required to disarm \(trap_i\), 
and \(0 \le C_{i} \le 10^{5}\), which denotes its complexity.\\ \\
The following conditions always hold: 
\begin{itemize}
    \item \(S_1\) = 1
    \item \(S_{1}\) \(<\) \(S_{2}\) \(<\) ... \(<\) \(S_{N-1}\) \(<\) \(S_N\)
    \item \(C_1\) \(>\) \(C_2\) \(>\) ... \(>\) \(C_{N-1}\) \(>\) \(C_N\)
    \item \(C_N\) = 0
\end{itemize}

\section*{Output}
% Describe the output. Kattis output judging is not sensitive to case changes or space changes (including line formatting and blank lines). Try to maintain this flexibility with your problem—do not be overly precise or prescriptive here in the output formatting description. However, if exact textual matching or floating-point precision is required, do specify the constraints (e.g., ``your answer should be correct within a relative or absolute error of $10^{-3}$'').
An integer $f$ denotes the minimum fatigue Jimothy can incur when disarming all the booby traps. \\

\noindent
\begin{minipage}[t]{0.48\textwidth}
    \section*{Sample Input 1}
    3 \\
    1 10 \\
    3 3 \\
    5 0 \\
\end{minipage}%
\hfill
\begin{minipage}[t]{0.48\textwidth}
    \section*{Sample Output 1}
    45
\end{minipage}
\begin{minipage}[t]{0.48\textwidth}
    \section*{Sample Input 2}
    3 \\
    1 10 \\
    10 5 \\
    11 0 \\
\end{minipage}%
\hfill
\begin{minipage}[t]{0.48\textwidth}
    \section*{Sample Output 2}
    110
\end{minipage}
\begin{minipage}[t]{0.48\textwidth}
    \section*{Sample Input 3}
    4\\
    1 10 \\
    4 6 \\
    25 5 \\
    26 0 \\
\end{minipage}%
\hfill
\begin{minipage}[t]{0.48\textwidth}
    \section*{Sample Output 3}
    196
\end{minipage}
\end{document}
